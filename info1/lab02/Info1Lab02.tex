\documentclass{instrukcja}
\usepackage[polish]{babel}
\usepackage[utf8]{inputenc}
\usepackage[OT4]{fontenc}

\begin{document}
\materialnumber{2}
\author{\L{}. \L{}aniewski-Wo\l{}\l{}k}
\course[Info I]{Informatyka I}
\material[Lab 2]{Laboratorium 2}
\materialtitle

\section{Funkcje}
Pewne zestawy operacji, zależne od zmiennych, możemy zebrać w grupki (funkcje) i wywoływać jak {\tt circle} i {\tt line}. Przykład z poprzedniego paragrafu możemy zamknąć w funkcji:
\begin{verbatim}
void obrazek(int h, int r)
{
    line( 10, 0, 0, h);
    line( 10, 0, 2*r, h);
    circle( 10+r, h, r);
}
\end{verbatim}

Pierwsza linia deklaruje funkcję, która jest zależna od dwóch parametrów: {\tt h,r}. Taką funkcję, możemy wywołać dla przykładu tak: {\tt obrazek(100,50);}. Spowoduje to wykonanie powyższych trzech operacji przy $h=100$ i $r=50$.

{\bf {\red Pamiętaj:} Nową funkcję napisz przed funkcją {\tt main}}

W funkcji {\tt main} wywołujemy funkcję {\tt obrazek}, tak jak {\tt circle} czy {\tt line}:
\begin{verbatim}
void main()
{
    graphics(200,200);
    obrazek(100,50);	
    wait();
}
\end{verbatim}

\subsection*{Ćwiczenia}
Napisz i wywołaj dowolne dwie z poniższych funkcji:
\begin{itemize}
\item {\tt prostokat(x,y,a,b)} --- Narysuje prostokąt o bokach {\tt a} i {\tt b} i środku w ({\tt x}, {\tt y})
\item {\tt kwadrat(x,y,r)} --- Narysuje kwadrat o boku $2r$ i wpisane koło o promieniu $r$.
\item {\tt ludzik(x,y,h)} --- Narysuje ludzika wysokości {\tt h} i środku głowy w ({\tt x}, {\tt y})
\item {\tt olimpiada(x,y)} --- Narysuje koła olimpijskie o środku w ({\tt x}, {\tt y})
\item[*] {\tt okno(a)} --- Używając funkcji do rysowania prostokąta narysuje okno o boku $a$.
\end{itemize}

\section{Trochę więcej szczegółów}
Omówmy pewne rzeczy trochę dokładniej.
\subsection{Typy}
W C i C++ \underline{musimy} deklarować zmienne, tzn. powiedzieć, jakich będziemy używać zmiennych i jakich one będą typów. Deklaracje piszemy 'typ zmienna1,zmienna2, ...;'. Najważniejsze typy to:
\begin{itemize}
\item {\tt int} --- Liczba całkowita (32-bitowa, od $-2^{31}$ do $2^{31}$)
\item {\tt float} --- Liczba zmienno-przecinkowa. Może opisywać ułamki dziesiętne z ok. 7 cyframi znaczącymi (32-bity)
\item {\tt double} --- Liczba zmienno-przecinkowa. Ma 16 cyfr znaczących (64-bity)
\end{itemize}

{\bf{\red Pamiętaj:} Jeśli używasz liczb rzeczywistych (a nie całkowitych), używaj typu {\tt double}.}

Pierwszym przykładem niech będzie:
\begin{verbatim}
double a;
a = 0;
while (a < 2*3.14)
{
    circle(a * 100, sin(a) * 100 + 100, 3);
    a = a + 0.001;
}
\end{verbatim}
Ten program narysuje wykres sinusa przeskalowany o 100, za pomocą kółek o promieniu 3.

\subsection*{Ćwiczenia}
Używając analogicznej pętli, wykonaj dowolne dwa z poniższych zadań.
\begin{itemize}
\item Narysuj wykres $a^2$.
\item Narysuj punkty o współrzędnych $x=100\sin{a}+100$ i $y=100\cos{a}+100$.
\item Narysuj punkty o współrzędnych $x=100\sin{a}\cos{4a}+100$ i $y=100\cos{a}\cos{4a}+100$.
\item Narysuj punkty o współrzędnych $x=100r\sin{a}+100$ i $y=100r\cos{a}+100$, gdzie $r = \frac{\cos{a}+2}{3}$ (niech $r$ będzie kolejną zmienną).
\end{itemize}

\subsection{Typy --- pułapki}
Ważne, by pamiętać, że liczby bez przecinka dziesiętnego, są uważane za całkowite, tzn. wykonywane są na nich działania jak dla liczb całkowitych. Dlatego {\tt 1/4} da jako wynik 0! Bo wynik 0.25 zostanie obcięty do liczby całkowitej. Żeby tego uniknąć, możemy napisać {\tt 1.0/4} lub jeszcze lepiej {\tt 1.0/4.0}. Możemy także bezpośrednio 'zrzutować' zmienne z {\tt int} na {\tt double} pisząc: {\tt (double) zmienna}.

{\bf{\red Pamiętaj:} Wszędzie, gdzie robisz obliczenia, używaj {\tt double}. Unikaj mieszania liczb całkowitych i zmienno-przecinkowych. Nigdy nie pisz ułamków jako 1/3}

\subsection*{Ćwiczenia}
Przeanalizuj (i przetestuj) wynik tego programu. Które linie nie dadzą pożądanego efektu?
\begin{verbatim}
double a;
a = 0;
while (a < 2)
{
    circle(a * 100, sin( a * 3.14 ) * 100 + 100, 3);
    circle(a * 100, sin( a * (314 / 100) ) * 100 + 100, 3);
    circle(a * 100, sin( (a * 314) / 100 ) * 100 + 100, 3);
    a = a + 0.001;
}
\end{verbatim}

\section{Funkcje po raz drugi}
Zestawy operacji, które powtarzamy w programie wielokrotnie, możemy zamknąć w funkcjach. Taka funkcja ,,połyka'' parametry i coś z nimi robi. Dla przykładu:
\begin{verbatim}
void kreski(int n, double r)
{
    int i;
    i = 0;
    while (i < n)
    {
        line(i, 0, i, r * i);
        i = i + 1;
    }
}
\end{verbatim}
W pierwszej linii mówimy:
\begin{itemize}
\item jak nazywa się funkcja --- {\tt kreski}
\item jakie ma parametry --- {\tt n} typu {\tt int} i {\tt r} typu {\tt double}
\item jakiego typu zwraca wartość ---w naszym wypadku {\tt void} oznacza, że nic nie zwraca
\end{itemize}

Gdy gdziekolwiek w funkcji {\tt main} użyjemy wywołania {\tt kreski( 20, 0.4);} jako efekt działania funkcji otrzymamy 20 pionowych kresek o długości od $0$ do $0.4\cdot 19$ (dlaczego $19$ a nie $20$?).

Taką funkcję możemy wykonać wielokrotnie dla różnych wartości $n$ i $r$:
\begin{verbatim}
void main()
{
    graphics(200,200);

    kreski( 10, 1.000);	
    kreski( 20, 0.500);	
    kreski( 30, 0.333);	
    kreski( 40, 0.250);	
	
    wait();
}
\end{verbatim}

\subsection*{Ćwiczenia}
Napisz i wywołaj dwie sposród niżej wymienionych funkcji.
\begin{itemize}
\item Funkcję, która narysuje ludzika wysokości {\tt h} i środku głowy w ({\tt x}, {\tt y}).
\item Funkcję, która w pętli narysuje tłum (używając poprzedniej funkcji).
\item Funkcję, która narysuje {\tt n} kółek w punkcie ({\tt x},{\tt y}) o coraz większych promieniach.
\item[*] Funkcję, która narysuje wielokąt foremny o {\tt n} bokach.
\end{itemize}


\section{Instrukcja warunkowa}
Kolejnym bloczkiem składowym programowania, jest instrukcja warunkowa. Sprawdza ona warunek i wykonuje pewną część kodu, tylko gdy warunek jest spełniony.
\begin{verbatim}
x = 2.0;
if ( x > 0 ) {
    y = sqrt(x);
} else {
    y = 0;
}
\end{verbatim}
Instrukcja ta sprawdza czy $x > 0$ i jeśli jest to prawdą, to wstawia $\sqrt{x}$ do zmiennej $y$. Gdy warunek nie jest spełniony, wykonywana jest część po {\tt else}, więc wstawiane jest $0$ do $y$. W ten sposób możemy zabezpieczyć się na przykład przed niemożliwymi obliczeniami, albo uzależnić działanie programu od jakiś wartości.

Zobaczmy prosty przykład:
\begin{verbatim}
double a;
a = 0;
while (a < 2*3.14)
{
    if (a < 2) {
        circle(sin(a) * 100 + 100, cos(a) * 100 + 100, 5);
    } else {
        circle(sin(a) * 100 + 100, cos(a) * 100 + 100, 10);
    }
    a = a + 0.001;
}
\end{verbatim}
Gdyby nie instrukcja {\tt if}, ten program narysował by koło z małych kółek. Teraz, gdy kąt {\tt a} przekroczy $2$ radiany zmieni promień kółeczka z $5$ na $10$ 

\subsection*{Ćwiczenia}
Napisz program który:
\begin{itemize}
\item Dla parametru $w$ rysuje wykres $x^2-w$, przeskalowany o 100 w obu kierunkach i przesunięty na środek (patrz poprzedni przykład).
\item Wyrysuje większe kółka w miejscach przecięcia wykresu z osią $x$ (\underline{jeżeli} przecina).
\item Zmodyfikuj program by działał dla dowolnych $a$, $b$, $c$ i funkcji $ax^2+bx+c$.
\end{itemize}

\end{document}
