\documentclass{instrukcja}        
\usepackage[polish]{babel}
\usepackage[utf8]{inputenc}
\usepackage[OT4]{fontenc}
%\usepackage{amsmath}

\begin{document}
\materialnumber{7}
\course[Info I]{Informatyka I}
\material[Lab 7]{Instrukcja 7}
\author{W. Regulski}
\materialtitle

\section*{Dynamiczna alokacja pamięci}
Do tej pory używaliśmy tablic, które miały z góry zadany rozmiar. Deklarowaliśmy je w bardzo prosty sposób:
\begin{verbatim}
double x[100];
\end{verbatim}

Oczywiście chcielibyśmy poszerzyć funkcjonalność naszego programu tak, aby była możliwa deklaracja  tablic o długości określonej już w trakcie działania programu. Mówimy wówczas o tzw. dynamicznej (tzn. wykonywanej podczas działania programu, a nie podczas kompilacji) alokacji pamięci. Taki mechanizm jest jak najbardziej dopuszczalny w języku C, składa się na niego kilka etapów:

\begin{enumerate}
\item Deklaracja wskaźnika do nowotworzonej tablicy.
\item Alokacja pamięci dla danej tablicy - używa się w tym celu funkcji {\tt malloc()}.
\item Zwolnienie pamięci - funkcja {\tt free()}.
\end{enumerate}

Nowe funkcje wymagają użycia biblioteki {\tt stdlib.h}.
Poniższy kawałek kodu ilustruje cały mechanizm:

\begin{verbatim}
#include <stdlib.h>

void main()
{
     int N=100;
     double *x;        // wskaźnik do pierwszego elementu tablicy
     
     x = (double*)malloc(N*sizeof(double));       // alokacja

     free(x);          // zwolnienie pamięci
}
\end{verbatim}

Kluczowym elementem całego mechanizmu jest właściwe zastosowanie funkcji {\tt malloc}. Jej argumentem jest rozmiar pamięci, o którą wnioskujemy. Potrzebujemy \(N\) razy rozmiar zajmowany przez zmienną typu {\tt double} - stąd obecność funkcji {\tt sizeof()}, która zwraca rozmiar danego typu zmiennych. Ponadto używamy mechanizmu rzutowania - przed funkcją {\tt malloc} pojawia się rzutowanie na typ {\tt double} poprzez wywołanie {\tt (double*)}\footnote{programując w czystym C można by teoretycznie zrezygnować z rzutowania, wówczas {\tt malloc} zwróci wskaźnik typu {\tt void*}. Jednak powszechnie stosujemy kompilatory języka C++, który jest językiem o tzw. silnej kontroli typów i nie zezwala na to, by wskaźnik nie miał typu. Nie zagłębiając się bardziej w szczegóły, należy po prostu wyrobić w sobie nawyk rzutowania przed użyciem funkcji {\tt malloc()}, aby uniknąć wielu nieprzyjemnych sytuacji w przyszłości. }. Po pracy na tablicy należy pamiętać o zwolnieniu pamięci, którą ona zajmowała. Stąd funkcja {\tt free()}.

\section*{Wczytywanie i rysowanie konturów}

Nowy mechanizm będzie nam potrzebny do wykonania zadania polegającego na wczytaniu współrzędnych konturów map z plików oraz narysowaniu tych map na ekranie. Współrzędne konturów dwóch map są zapisane w dwóch plikach, {\tt plik1.txt} oraz {\tt plik2.txt}. Pliki mają następującą strukturę:
\begin{verbatim}
156
12.67	 768.3254
14.98	 768.3254
17.462	766.51075
...
\end{verbatim}
Pierwsza linia pliku zawiera ilość punktów zapisanych w pliku. Poniżej współrzędne \(x\) oraz \(y\) są zapisane w oddzielnych kolumnach. 

\subsection*{Ćwiczenia}
\begin{enumerate}
\item Otwórz oba pliki w programie głównym.
\item Wczytaj pierwsze linie obu plików do programu. Wydrukuj na ekran te wartości, aby upewnić się, że wykonano ten podpunkt bezbłędnie.
\item Użyj wczytanych wartości do alokacji tablic, które będą przechowywać wspólrzędne punktów. Użyj mechanizmu alokacji dynamicznej.
\item Wczytaj współrzędne z obu plików do wcześniej zadeklarowanych tablic. Upewnij się, czy wykonano to dobrze.
\item Napisz funkcję {\tt void PrintCoords(double *x, double *y, int N,int start, int end)}, która będzie drukować na ekranie zadany przedział współrzędnych z danej mapy. Funkcja ma informować, gdy zadany przedział nie może być wydrukowany (bo np. kres górny przekracza długość tablicy)
\item napisz funkcję {\tt void Display()}, która będzie rysować zadany kontur mapy na ekranie w oknie graficznym. Kontur ma być narysowany za pomocą linii, które będą łączyć kolejne punkty. Zwroć uwagę na to, by połączyć również ostatni punkt z pierwszym. Wyświetl oba kontury na jednym obrazku. Co to za mapy?
\end{enumerate}

\section*{Trochę geografii}

Mając już kontur, możemy policzyć kilka interesujących rzeczy:

\subsection*{Ćwiczenia}
\begin{enumerate}
\item Napisz funkcję {\tt double Perimeter()}, która zwróci obwód zadanego konturu.
\item Napisz funkcję {\tt double Area()}, która policzy oraz zwróci powierzchnię danej mapy. Powierzchnię mapy należy policzyć jako sumę powierzchni trójkątów, których podstawy są kolejnymi odcinkmi konturu a wierzchołek jest zlokalizowany gdziekolwiek (najwygodzniej położyć go w zerze). Pole takiego trójkąta będzie po prostu połową modułu iloczynu wektorowego, gdzie wektorami są boki trójkąta:
\begin{displaymath}
S_{i}=\frac{1}{2} | \boldsymbol v_i\times \boldsymbol v_{i+1} | = \frac{1}{2} (x_iy_{i+1}-x_{i+1}y_i)
\end{displaymath}
Pamiętaj o ostatnim i pierwszym punkcie. Zwróć uwagę, że otrzymane pole może być ujemne. Od czego to zależy?
\item Policz rozciągłość południkową i równoleżnikową obu obszarów. Zaznacz najbardziej skrajne punkty na obu mapach za pomocą kółka.
%\item *Zakreskuj obszar wspólny obu map\footnote{Ten podpunkt wymaga sporo tzw. dłubaniny}.
\end{enumerate}

\end{document}
