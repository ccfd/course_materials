\documentclass{instrukcja}        
\usepackage[polish]{babel}
\usepackage[utf8]{inputenc}
\usepackage[OT4]{fontenc}

\begin{document}

\materialnumber{8}
\course[Info I]{Informatyka I}
\material[Lab 8]{Instrukcja 8}
\author{B. Górecki}
\materialtitle

\section*{Dynamiczna alokacja tablic wielowymiarowych}
Tydzień temu nauczyliśmy się dynamicznej alokacji pamięci dla tablic jednowymiarowych. Dynamiczna oznacza tyle, że rozmiar tablicy, jaką chcemy zaalokować, znamy dopiero w momencie wykonywania programu, tzn., że nie jesteśmy w stanie go określić statycznie (daną, konkretną liczbą) na etapie kompilacji. Kod do dynamicznej alokacji tablic jednowymiarowych wyglądał tak:
\begin{verbatim}
#include <stdlib.h>

void main()
{
   double *tab;
   int n;
   printf("Podaj n:\n");
   scanf("%d, &n);

   tab = (double*)malloc(n*sizeof(double));
   // Tu mozesz wykonywac operacje na tablicy
   free(tab);
}
\end{verbatim}
Dziś nauczymy się dwóch nowych rzeczy: stosowania w języku C tablic wielowymiarowych (alokowanych statycznie) oraz ich dynamicznej alokacji.

\section*{Tablice wielowymiarowe}
Język C pozwala na stosowanie tablic wielowymiarowych. Do tej pory przez kilka tygodni używaliśmy jedynie tablic jednowymiarowych. Potocznie często określaliśmy je {\it wektorami}. Wyobraźmy sobie - tablica dwuwymiarowa doskonale nadaje się np. do przechowywania macierzy.\footnote{Tak naprawdę wiele więcej struktur - często nawet zupełnie niematematycznych możemy trzymać w dwuwymiarowych tablicach - np. programując grę w szachy moglibyśmy użyć dwuwymiarowej tablicy o wymiarze 8 x 8 i w odpowiednie pola tej tablicy wpisywać liczby, które symbolizowałyby konkretne figurę. A gra w statki? Można podobnie. Z drugiej strony w praktycznych obliczeniach numerycznych wielkie macierze o rozmiarze rzędu kilkuset tysięcy do kilku milionów i większe przechowuje się w postaci wektora. W praktycznych zagadnieniach są to niemal zawsze macierze rzadkie, tzn. takie, które w stosunku do całkowitej liczby swoich elementów mają bardzo niewiele elementów, które nie są zerem. Taką macierz łatwo jest trzymać w pamięci jako wektor (przyjmując specjalny format, który pomija wszystkie zera - np. tzw. format CSR {\it (ang. compressed sparse row)} jest szeroko stosowanym formatem zapisania macierzy rzadkiej w trzech wektorach). Tak wielka macierz przechowywana jawnie najpewniej nie zmieściłaby się w pamięci żadnego dostępnego nam komputera.} Przyjrzyjmy się więc fragmentowi kodu, który zadeklaruje dwuwymiarową tablicę o wymiarze 3x4. Możemy to utożsamić z reprezentacją macierzy o takim samym wymiarze.
\begin{verbatim}
void main()
{
   double A[3][4];
   
   // Tu mozemy przypisac wartosci kolejnym elementom:
   A[0][0] = 1;
   A[0][1] = 1.5;
   A[0][2] = 0;
   A[0][3] = -2.7;
   ...
   A[2][3] = 8;
   // Tu mamy wypelniona macierz
   // Mozemy wykonywac obliczenia
}
\end{verbatim}
Zauważmy, że w przypadku tablic deklarowanych statycznie nie ma potrzeby ich zwalniania. Kompilator sam o to dba (jak w przypadku wszystkich zmiennych, które do tej pory deklarowaliśmy - one też są automatycznie niszczone przez kompilator). Powyższą macierz możemy też wypełnić wartościami w nieco zgrabniejszy sposób niż przez wypisanie kolejnych dwunastu linijek przypisań. Możemy to zrobić na liście inicjalizacyjnej od razu na etapie deklaracji tablicy. Dokonuje się tego tak:
\begin{verbatim}
double A[3][4] = {{1, 1.5, 0, -2.7}, {-3, 2.5, 7, 0}, {0, 1, -3, 8}};
\end{verbatim}
W ten sposób stworzymy poniższą macierz:
\begin{equation}
\mathbf{A} = 
\left(
\begin{array}{cccc}
1 & 1.5 & 0 & -2.7 \\ -3 & 2.5 & 7 & 0 \\ 0 & 1 & -3 & 8
\end{array}
\right)
\end{equation}
Pozostaje wytłumaczyć jeszcze, w jaki sposób odwołujemy się do elementów w dwuwymiarowej tablicy. Robimy to analogicznie do tablicy jednowymiarowej, tylko tym razem musimy podać dwa indeksy. Tak więc do elementu macierzy $a_{32}$ odwołamy się przez napisanie {\tt a[2][1]}. W ten sposób możemy wyłuskać wartość przechowywaną pod tym elementem lub operatorem {\it =} przypisać temu elementowi nową wartość.

\section*{Ćwiczenia}
W funkcji {\tt main} napisz fragment kodu, w którym zadeklarujesz i zainicjalizujesz dowolnymi wartościami dwie różne tablice dwuwymiarowe. Jedna ma przechowywać macierz kwadratową o wymiarze 2, a druga z nich macierz kwadratową o wymiarze 3. Napisz kod, który dla każdej z tych macierzy policzy wyznacznik.

\section*{Dynamiczna alokacja}
Czas jednak na dynamiczną alokację. Dwuwymiarową tablicę o rozmiarze M x N zaalokujemy w następujący sposób: stworzymy tablicę jednowymiarową o rozmiarze M (umówmy się, że ona będzie wskazywać na początek każdego z wierszy), po czym każdemu z elementów tej tablicy zaalokujemy blok o długości N (to będą jednowymiarowe tablice do przechowywania kolejnych wierszy). De facto będziemy mieli w pamięci M bloków, każdy długości N. Spójrzmy na kod.
\begin{verbatim}
double **A;
*A = (double**)malloc(M*sizeof(double*));
\end{verbatim}
Zauważmy, że tydzień temu alokowaliśmy jednowymiarową tablicę jako wskaźnik. Tym razem będziemy mieć tablicę dwuwymiarową, więc używamy podwójnego wskaźnika. Dlatego w instrukcji powyżej blok pamięci zwracany przez funkcję {\tt malloc} rzutujemy na podwójny wskaźnik {\tt double**}. Musimy też obliczyć, ile miejsca potrzebujemy. Przechowywać będziemy wskaźniki (do odpowiednich tablic jednowymiarowych przechowujących wiersze), dlatego jako argument funkcji {\tt sizeof} podajemy {\tt double*}. Teraz alokujemy tablice jednowymiarowe do przechowywania wierszy.
\begin{verbatim}
for(int i = 0; i<M; ++i)
   A[i] = (double*)malloc(N*sizeof(double));
\end{verbatim}
Powyżej każdemu z elementów pierwszej tablicy przypisaliśmy tablicę do przechowywania każdego z wierszy. Tym razem blok zwrócony przez funkcję {\tt malloc} rzutujemy na typ {\tt double*}, a argumentem funkcji {\tt sizeof} jest typ zmiennej przechowywanej w tej tablicy, czyli już zwykła zmienna {\tt double}, a nie wskaźnik do niej. Tablica dwuwymiarowa jest już gotowa - możemy jej używać.

Po zakończeniu pracy z tablicą, trzeba koniecznie zwolnić pamięć przez nią wykorzystywaną. Teraz trzeba operacje wykonać od końca! Tzn. najpierw zwalniamy każdy z wierszy, a na końcu zwolnimy pierwotną tablicę wskaźników do wierszy. Dokonuje tego poniższy kod.
\begin{verbatim}
for(int i = 0; i<M; ++i)
   free(A[i]);

free(A);
\end{verbatim}

\subsection*{Podsumowanie}
Zbierzmy wszystkie instrukcje w jednym miejscu. Chcemy zaalokować dwuwymiarową tablicę zmiennych typu {\tt int}. Dokonujemy tego tak:
\begin{verbatim}
// Alokacja pamieci
int **A;
*A = (int**)malloc(M*sizeof(int*);
for(int i = 0; i<M; ++i)
   A[i] = (int*)malloc(N*sizeof(int));

// Tu mozemy wykonywac dowolne operacje na tablicy

// Zwolnienie pamieci
for(int i = 0; i<M; ++i)
   free(A[i]);
free(A);
\end{verbatim}

\section*{Ćwiczenia}
\begin{enumerate}
\item Napisz funkcję, która jako argumenty przyjmie podwójny wskaźnik (wskaźnik do dynamicznie alokowanej tablicy dwuwymiarowej) oraz liczbę kolumn i wierszy macierzy, po czym dokona wydruku macierzy na ekran w naturalnej postaci, do jakiej jesteśmy przyzwyczajeni dla macierzy.
\item Napisz funkcję {\tt maxAbsAij}, która dla danej macierzy zwróci do funkcji {\tt main} największy co do modułu element tej macierzy oraz jego indeksy $i$, $j$.
\item Napisz funkcję {\tt minAbsAij}, która dla danej macierzy zwróci do funkcji {\tt main} najmniejszy co do modułu element tej macierzy oraz jego indeksy $i$, $j$.
\item W funkcji main zaalokuj w sposób dynamiczny miejsce dla macierzy 3x3 oraz dwóch wektorów trzyelementowych. Wypełnij macierz i jeden z wektorów dowolnymi wartościami, po czym napisz w funkcji {\tt main} kod, który dokona przemnożenia danej macierzy przez dany wektor i wynik mnożenia wpisze do drugiego z wektorów. Mnożenie macierzy przez wektor określone jest wzorem \begin{equation}w_i = \sum\limits_{j=0}^{n-1}{a_{ij}v_j}\end{equation}.
\item Zamknij powyższe operacje w funkcji o nagłówku \begin{verbatim}void MatVecMultiply(double **A, double *v1, double *v2, int n)\end{verbatim} i dokonaj wywołania z funkcji {\tt main}.
\item Zmodyfikuj powyższy program tak, żeby rozmiar {\it n} był wczytywany z klawiatury, elementy tablicy były generowane zgodnie ze wzorem $a_{ij} = \frac{i+1}{j+1}, (i,j = 0,...,n-1)$, zaś elementy wektora wg wzoru $v_i = i+1, (i = 0,...,n-1)$. Obliczaj iloczyn takiej macierzy przez ten wektor, korzystając ze swojej funkcji {\tt MatVecMultiply}. Wynik wyświetlaj na ekranie oraz sprawdź, czy otrzymujesz poprawny wynik.\footnote{Dla takiej macierzy i takiego wektora bardzo łatwo jest wygenerować analityczny wynik. Wypisz sobie małą macierz wg zadanego wzoru i odpowiadający wektor i na pewno szybko zauważysz prawidłowość. Będziesz wiedzieć, jaki wynik powinien dać program. Tak się testuje programy na wczesnych etapach rozwoju.}
\item Napisz funkcję \begin{verbatim}MatMatMultiply(double **A, double **B, double **C, int mA,
                                     int nA, int mB, int nB)\end{verbatim}służącą do mnożenia dwóch macierzy prostokątnych ($\bf A$ o wymarze $m_A \times n_A$ i $\bf B$ o wymiarze $m_B \times n_B$) i wpisującą wynik do macierzy $\bf C$ (zadbaj w funkcji {\tt main} o to, aby pamięć zaalokowana dla macierzy $\bf C$ była odpowiedniej wielkości - zgodnej z regułami mnożenia macierzy). {\bf Uwaga:} Pamiętaj, aby koniecznie zwolnić wszelką dynamicznie alokowaną pamięć. \footnote{Niezwolnienie pamięci prowadzi do jej wycieków i w przypadku pewnych operacji wykonywanych w pętlach może doprowadzić do tego, że Twój program wykorzysta całą pamięć operacyjną komputera i przestanie działać.}
\end{enumerate}
\section*{Dla ambitnych*}
Zastanów się, jak wyglądałaby dynamiczna alokacja i zwolnienie pamięci dla tablicy trójwymiarowej. Na przykład, gdybyś chciał napisać grę w trójwymiarowe kółko i krzyżyk. Skonsultuj z prowadzącym kod dla takiego przypadku.


\end{document}
