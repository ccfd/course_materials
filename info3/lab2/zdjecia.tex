\documentclass{instrukcja}
\usepackage[polish]{babel}
\usepackage[utf8]{inputenc}
\usepackage[OT4]{fontenc}

\begin{document}

\materialnumber{2}
\course[Info I]{Informatyka I}
\material[Lab 2]{Instrukcja 2}
\author{Ł. Łaniewski-Wołłk}
\materialtitle

\section{Materiały}

Materiały do tego laboratorium można znaleźć w katalogu {\tt \textasciitilde{}llaniewski/lab2\_gorne } lub  {\tt \textasciitilde{}llaniewski/lab2\_dolne }  na {\tt info3.meil.pw.edu.pl} ( górne: sala 120, dolne: sala 20B ). 

Pod linuxem należy w terminalu wykonać:
\begin{verbatim}
> scp [login]@orange.meil.pw.edu.pl:~llaniewski/lab2_gorne .
> ./lab2_gorne [login na info3.meil.pw.edu.pl]
\end{verbatim}
Następnie wyłączyć terminal i włączyć nowy. Polecenia te zgrają przygotowaną wersję programu ImageMagick i zainstalują, a także utworzą katalog {\tt LAB2} w katalogu domowym, w który zawarte są przykładowe obrazki do obróbki.

\section{BASH: skrypty}
Pisanie skryptów, polega na spisaniu w pliku komend, które normalnie wpisywalibyśmy w linii poleceń. Taki plik możemy następnie oznaczyć jako wykonywalny komendą {\tt chmod +x plik} i wykonać komendą {\tt ./plik}. Linia poleceń (BASH) służy do uruchomiania programów --- dlatego:\\{\bf każda linijka skryptu wygląda nastepująco:} ,,{\tt {\red program} {\green agumenty}}''.

Przeanalizuj fragment kodu, z zaznaczonymi {\red programami} i {\green opcjami}:
{\tt\\
{\red i=1}\\
while {\red test} {\green \$i -lt 10}\\
do\\
{\red echo} {\green \$i}\\
{\red cp} {\green plik plik\_\$i}\\
{\red i=}\$({\red expr} {\green \$i + 1})\\
done\\
}

Gdy zapamiętamy tą zasadę, łatwo zobaczyć, że:
\begin{itemize}
\item {\tt\red i=1} piszemy bez spacji ponieważ wtedy BASH wie, że to przypisanie, a nie program {\tt\red i} z opcjami {\tt\green =} i {\tt\green 1}. 
\item w wyrażeniu {\tt {\red expr} {\green \$i + 1}}, musimy zachować spacje, żeby program {\tt\red expr} dostał trzy argumenty ,,{\tt\green\$i}'', ,,{\tt\green +}'' i ,,{\tt\green 1}'', a nie jeden ,,{\tt\green i+1}''.
\item w pętli {\tt while}, nie możemy wpisać ,,{\tt i<10}'', lecz musimy użyć jakiegoś programu. Do wszelkiego rodzaju testów stwożony został program {\tt\red test}. W tym wypadku podajemy mu za argumenty ,,{\tt\green\$i}'', ,,{\tt\green -lt}'' i ,,{\tt\green 10}'', gdzie opcja {\tt\green -lt} oznacza ,,less than''.
\end{itemize}

\subsection{Przydatne programy}
Jeśli już wiemy, że każdy skrypt w BASH to seria wywołanych programów, to potrzebne jest nam dużo małych programów, z których będziemy mogli tworzyć skrypty.
\begin{itemize}
\item {\tt{\red echo} {\green tekst}} --- Wpisuje {\green tekst} na ekran.
\item {\tt{\red cat} {\green plik}} --- Wypisuje zawartość {\green plik}u na ekran
\item {\tt{\red grep} {\green tekst}} --- Czyta z klawiatury tekst i wypisuje tylko linie zawierające {\green tekst}
\item {\tt{\red grep} {\green tekst pliki}} --- Wyszukuje {\green tekst} w {\green plik}ach
\item {\tt{\red cd} {\green katalog}} --- Wchodzi do {\green katalogu}
\item {\tt{\red ls} {\green katalog}} --- Wypisuje zawartość {\green katalog}u na ekran
\item {\tt{\red cp} {\green pliki katalog}} --- Kopiuje {\green pliki} do {\green katalog}u
\item {\tt{\red cp} {\green plik1 plik2}} --- Kopiuje plik o nazwie {\green plik1} do pliku o nazwie {\green plik2}
\item {\tt{\red mv} {\green pliki katalog}} --- Przenosi {\green pliki} do {\green katalog}u
\item {\tt{\red mv} {\green plik1 plik2}} --- Zmienia nazwę pliku z {\green plik1} na {\green plik2}
\item {\tt{\red sed} {\green 's/tekst1/tekst2/g'}} --- Czyta z klawiatury tekst i go wypisuje zamieniając ,,{\green tekst1}'' na ,,{\green tekst2}''
\end{itemize}

\subsection{Przekierowanie wejścia wyjścia}
Standardowo wszystkie programy czytają z klawiatury i piszą na ekran. Można jednak zarówno pierwsze jak i drugie przekierować.
\begin{itemize}
\item {\tt {\red program} {\blue > plik}} --- To co program wypisałby na ekran, zostanie wpisane do {\blue plik}u ({\blue plik} zostanie nadpisany jeśli istnieje)
\item {\tt {\red program} {\blue >> plik}} --- To co program wypisałby na ekran, zostanie \uline{dopisane} do {\blue plik}u ({\blue plik} zostanie utwożony jeśli nie istniał)
\item {\tt {\red program} {\blue < plik}} --- Program dostanie zawartość {\blue plik}u, tak jakbyśmy ją wpisali z klawiatury
\item {\tt {\red program1} {\blue |} {\red program2}} --- To co {\red program1} wypisałby na ekran, zostanie wpisane ,,z klawiatury'' do {\red program2}
\item {\tt {\blue `}{\red program}{\blue `}} lub {\tt {\blue \$(}{\red program}{\blue )}} --- To co program wypisałby na ekran, zostanie \uline{wklejone} w tym miejscu kodu (patrz przykłady). \uline{Znak {\tt\blue `} jest na klawiaturze przy tyldzie {\tt\textasciitilde{}}.}
\end{itemize}

Przykłady:
\begin{itemize}
\item {\tt echo Tekst {\blue > plik}} --- wypisze ,,Tekst'' do {\blue plik}u ({\blue plik} zostanie nadpisany jeśli istnieje)
\item {\tt echo Tekst {\blue >> plik}} --- dopisze ,,Tekst'' do {\blue plik}u ({\blue plik} zostanie utwożony jeśli nie istniał)
\item {\tt grep Tekst {\blue < plik}} --- wyszuka w {\blue plik}u linie zwierające {\green ,,Tekst''} i je wypisze na ekran
\item {\tt echo Tekst {\blue |} sed 's/st/a/g'} --- Zamieni w ,,Tekst'' każde wystąpienie ,,st'' na ,,a''. Więc wypisze na ekran ,,Teka''.
\item {\tt echo \$nazwa {\blue |} sed 's/\textbackslash{}.txt/.dat/g'} --- Zastąpi w zmiennej {\tt nazwa} końcówkę {\tt .txt} na {\tt .dat}. \uline{Rezultat wypisze na ekran}.
\item {\tt echo \$nazwa {\blue |} sed 's/\textbackslash{}.txt/.dat/g'} --- Zastąpi w zmiennej {\tt nazwa} końcówkę {\tt .txt} na {\tt .dat}. \uline{Rezultat wypisze na ekran}.
\item {\tt nazwa2={\blue \$(}echo \$nazwa {\blue |} sed 's/\textbackslash{}.txt/.dat/g'{\blue )}} --- Jak poprzednio, lecz \uline{rezultat wypisze do zmiennej} {\tt nazwa2}.
\item {\tt ls katalog {\blue > plik}} --- wypisze zawartość {\green katalog}u do {\blue plik}u ({\blue plik} zostanie nadpisany jeśli istnieje)
\item {\tt cp {\blue `}ls{\blue `} katalog} albo {\tt cp {\blue \$(}ls{\blue )} katalog} --- skopiuje pliki do {\tt katalog}u według listy zwróconej przez {\tt ls}.
\item {\tt cp {\blue `}cat plik{\blue `} katalog} bądz {\tt cp {\blue \$(}cat plik{\blue )} katalog} --- skopiuje pliki do {\tt katalog}u według listy zawartej w {\tt plik}u.
\end{itemize}
\subsection{Pętle i wyrażenia warunkowe}
\begin{itemize}
\item {\tt if {\red program} {\green argumenty}\\
then\\
{\red polecenia1}\\
else\\
{\red polecenia2}\\
fi}\\
Jeśli wykonanie ,,{\red program} {\green argumenty}'' się powiedzie (program zwróci $0$), to wykonane zostaną {\red polecenia1}. W przeciwnym wypadku wykonane zostaną {\red polecenia2}.
\item {\tt while {\red program} {\green argumenty}\\
do\\
{\red polecenia}\\
done}\\
Pętla, która będzie wykonywać {\red polecenia}, puki ,,{\red program} {\green argumenty}'' będzie wykonywany z powodzeniem.
\item {\tt for i in {\green lista}\\
do\\
{\red polecenia}\\
done}\\
Pętla, która po kolei każdy element {\green list}y wstawi do zmiennej {\tt i}, a następnie wykona {\red polecenia}.
\end{itemize}

Dla przykładu:\\
{\tt for i in {\green *.jpg}\\
do\\
{\red mv} {\green \$i IMG/a\_\$i} \\
done}\\
Przeniesie każdy plik o końcówce {\tt .jpg}, do katalogu {\tt IMG} dodając im przedrostek {\tt a\_} (np.: {\tt obrazek.jpg} zamieni na {\tt IMG/a\_obrazek.jpg}).

\subsection{Ćwiczenia}

Domyślnym edytorem na serwerze {\tt info3 } jest edytor {\tt nano }, dostępny jest teś edytor {\tt vim }. Pierwszy z nich wydaje się prostszy w obsłudze, drugi występuje na prawie każdym komputerze z UNIXem.

\begin{itemize}
	\item Przy pomocy pętli wypisz na ekran liczby od 0 do 10
	\item Zmień skrypt, tak aby wypisywał od 0 do podanej jako argument wielkości
\end{itemize}


\section{Obróbka obrazków}
\subsection{{\tt convert}}
Głównym programem którego będziemy używać to {\tt convert} z biblioteki ImageMagick. Program ten służy do najróżniejszego typu konwersji i zmiany właściwości obrazów --- lecz potrafi także dodawać elementy do obrazu, a nawet tworzyć obrazy od zera. Najłatwiej zobaczyć jego użycie na przykładach:

{\bf UWAGA: Zanim zaczniesz, skopiuj katalog ze zdjęciami do jakiegoś tymczasowego katalogu!}

\begin{itemize}
\item {\tt convert plik.gif plik.jpg} --- przekonwertuje plik w formacie GIF na format JPEG
\item {\tt convert plik1.jpg -resize 50\% plik2.jpg} --- zmniejszy obrazek dwukrotnie
\item {\tt convert plik1.jpg -resize 100 plik2.jpg} --- zmniejszy obrazek, tak by krótszy wymiar był 100 pikseli
\item {\tt convert plik1.jpg -resize 100x100 plik2.jpg} --- zmniejszy obrazek tak, by mieścił się w kwadracie 100 na 100 pikseli
\item {\tt convert plik1.jpg -resize 100x100\textbackslash{}! plik2.jpg} --- zmniejszy obrazek \uline{dokładnie} do rozmiaru 100 na 100 pixeli
\item {\tt convert -size 320x85 canvas:none -font Bookman-DemiItalic -pointsize 72 -draw "text 25,60 'Magick'"{ }-channel RGBA -blur 0x6 -fill darkred -stroke magenta -draw "text 20,55 'Magick'"{ }fuzzy-magick.jpg} --- stworzy obrazek fuzzy-magick.jpg, z tekstem "Magick"
\end{itemize}
Wykonaj powyższe operacja, sprawdź efekty.


\subsection*{Ćwiczenia}
Napisz skrypt który:
\begin{itemize}

\item Zmniejszy wszystkie pliki {\tt jpg}
\item Napisz skrypt który: Zmniejszy wszystkie pliki {\tt jpg} umieszczając je w innym katalogu
\item Napisz skrypt który: Skonwertuje wszystkie pliki {\tt jpg} na {\tt gif}, dodając końcówkę: {\tt plik.jpg $\rightarrow$ plik.jpg.gif}
\item Napisz skrypt który: Skonwertuje wszystkie pliki {\tt jpg} na {\tt gif}, zamieniając końcówkę {\tt plik.jpg $\rightarrow$ plik.gif}
\item Na każde zdjęcie naniesie tekst używając {\tt -pointsize {\green rozmiar} -draw "{}text {\green x},{\green y} '{\green Tekst}'"{}}
\item Na każde zdjęcie naniesie aktualną datę (komenda {\tt date})
\item Na każde zdjęcie naniesie datę utworzenia tego zdjęcia (można ją wyciągnąć przy pomocy {\tt stat -c \%y plik})
\item Zmniejszy wszystkie obrazki z katalogu {\tt drop1} i połączy je w animację przy pomocy {\tt convert *.jpg animacja.gif}
\end{itemize}


\end{document}
