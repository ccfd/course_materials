\documentclass{instrukcja}
\usepackage[polish]{babel}
\usepackage[utf8]{inputenc}
\usepackage[OT4]{fontenc}

%\usepackage{dsfont}
%\usepackage[usenames,dvipsnames,svgnames,table]{xcolor}
%\usepackage[justification=centering]{caption}

\begin{document}

\materialnumber{1}
\course[MetNum]{Metody Numeryczne}
\material[Lab 1]{Instrukcja 1}
\author{Ł. Łaniewski-Wołłk}
\materialtitle
%\title{Optymalizacja --- Adjoint}

Na dzisiejszych zajęciach dobierzemy wektor grubości elementów $\theta=\text{\tt thick}$, tak by ugięcie belki było jak najmniejsze. Jedynym elementem który zależy od $\theta$ jest macierz sztywności:

\[S(\theta)x = F\]

A dokładniej: macierz sztywności jest sumą macierzy elementów przemnożonych przez elementy $\theta$. Dla późniejszej poprawy zbieżności dołożymy potęgę $\gamma=3$ do tej zależności:
\[S(\theta) = \sum_i\theta_k^\gamma K^k\]

Będziemy chcieli zoptymalizować przesunięcie węzła w którym przyłożyliśmy siłę. Stwórzmy wektor $g$, który ma $-1$ w miejscu tego przemieszczenia, a resztę wartości ma zerową. Teraz nasza funkcja celu to $J = -\langle x, g \rangle$. Potraktujmy nasze równanie statyki jako więz i rozpiszmy funkcję celu powiększoną o mnożniki Lagrange:
\[-\sum_i x_i g_i  + \sum_j\lambda_j (\sum_iS_{ji}(\theta)x_i - F_j)\]
Optymalne rozwiązanie powinno zerować pochodne po $x$, $\lambda$ i $\theta$:
\[\begin{cases}
-g_i + \sum_j\lambda_j S_{ji} (\theta) &= 0\\
\sum_i S_{ji}(\theta)x_i - F_j &= 0\\
\sum_{ij}\lambda_j \pr{\theta_k}S_{ji}(\theta)x_i &= 0
\end{cases}\]

Drugie równanie to nasze równanie statyki. Pierwsze równanie to równanie sprzężone (adjoint):
\[S^T(\theta)\lambda = g\]
Trzecie równanie to równanie ma gradient funkcji celu względem parametrów:
\[\frac{d}{d\theta_k}J = \sum_{ij}\lambda_j \pr{\theta_k}S_{ji}(\theta)x_i = \gamma\theta_k^{\gamma-1}\sum_{ij}K^k_{ji}(\theta)\lambda_jx_i\]

\begin{zad}
Wprowadź w funkcji mnożącej przez macierz sztywności {\tt SMult} parametr $\gamma=3$ ({\tt gamma}) podnosząc {\tt thick} do potęgi {\tt gamma}. Zdefiniuj zmienną $\text{\tt frac} = 0.5$ i ustaw początkowe $\theta$ ({\tt thick}) na równe {\tt frac}. 
\end{zad}

\begin{zad}
Zdefiniuj wektor $g$ i rozwiąż równanie sprzężone (zauważ że $S$ jest symetryczna).
\end{zad}

\begin{zad}
Zdefiniuj funkcję {\tt calc\_grad(int n, double * x, double * lambda, double * grad)}. Skopiuj do niej zawartość funkcji mnożącej {\tt SMult} i zmień:\\
{\tt r[$\diamondsuit$] += x[$\spadesuit$]*pow(thick[$\clubsuit$],gamma)*$\heartsuit$;}\\
na:\\
{\tt grad[$\clubsuit$] += gamma*pow(thick[$\clubsuit$],gamma-1)*lambda[$\diamondsuit$]*x[$\spadesuit$]*$\heartsuit$;}\\
Wyświetl tak policzony gradient. Pamiętaj, że gradient ma taką samą długość jak {\tt thick}, czyli {\tt mx*my}. Pamiętaj także by wyzerować {\tt grad} i wyciąć część murującą stopnie swobody.
\end{zad}


\section{Optymalizacja}

Gradient wskazuje nam w jakim kierunku powinniśmy przesuwać nasze wartości parametrów by uzyskać lepszy wynik. Pierwszym nasuwającym się schematem postępowania byłoby:\\
{\tt thick[i] += grad[i]; }
\begin{zad}
Dodaj gradient do parametrów {\tt thick[i] += grad[i];}. Iteruj taką procedurę, oglądając wyniki.
\end{zad}

Tak ustawiony problem optymalizacyjny jest nieograniczony. Chcemy jednak uzyskać najmniejsze ugięcie przy ustalonej ,,masie'' belki. Tzn: chcemy zachować sumę parametrów $\theta$: $\sum_i\text{\tt thick[i]} = \text{\tt frac*mx*my}$. Możemy łatwo nałożyć ten więz na {\tt grad}:

\begin{zad}
Odejmij od wektora {\tt grad} jego sumę.
\end{zad}

W kolejnych iteracjach grad ma różną skalę. Na początku jest duży, a później mały. Typową techniką w takich wypadkach jest normalizacja:

\begin{zad}
Zdefiniuj zmienną {\tt move = 0.05}. Podziel {\tt grad} przez jego największy element i pomnóż przez {\tt move*5}.
\end{zad}

Na nasz projekt musimy jednak narzucić bardziej istotne warunki. Po pierwsze nigdzie grubość nie może przekroczyć $1$, i musi być powyżej $0$. Ponadto zazwyczaj chcemy, by zmiana w pojedynczej iteracji nie przekroczyła {\tt move}. Te warunki dość trudno pogodzić z warunkiem stałej sumy elementów. 

\begin{zad}
Wynik dodania gradientu do parametrów wstaw do nowego wektora {\tt nt[i] = thick[i] + grad[i];}. Dla danego parametru {\tt scale} oblicz {\tt thick[i] = scale * nt[i];}. Na tak obliczone {\tt thick} narzuć powyżej opisane 4 warunki, obcinając za duże, bądź za małe wartości.
\end{zad}

\begin{zad}
Zsumuj wartości {\tt thick} po poprzedniej procedurze. Dobierz {\tt scale} metodą bisekcji tak by $\sum_i\text{\tt thick[i]} = \text{\tt frac*mx*my}$.
\end{zad}

\begin{zad}
Przetestuj program dla różnych obciążeń, ustawień parametru {\tt move} i ustawień maksymalnej liczby iteracji w metodze gradientów sprzężonych.
\end{zad}

\end{document}
